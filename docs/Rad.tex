\documentclass[]{foi} % zakomentirati za pisanje rada na engleskom jeziku
% \documentclass[english]{foi} % odkomentirati za pisanje rada na engleskom jeziku
\usepackage[utf8]{inputenc}
\usepackage{lipsum}

\vrstaRada{\projekt}
% \zavrsni ili \diplomski ili \seminar ili \projekt

\title{Implementacija Web poslužitelja korištenjem funkcijskog pristupa}
\predmet{\predmetDP}
% ostaviti prazno ako \vrstaRada nije \projekt ili \seminar
% \predmetBP ili \predmetDP ili \predmetTBP ili \predmetVAS

\author{Hrvoje Lesar} % ime i prezime studenta/studentice
\spolStudenta{\musko} % \zensko ili \musko

\mentor{Bogdan Okreša Đurić} % ime i prezime mentora
\spolMentora{\musko} % \zensko ili \musko
\titulaProfesora{dr. sc.}
% HR: dr. sc.  / doc. dr. sc. / izv. prof. dr. sc. / prof. dr. sc. 
% EN: -prazno- / Asst. Prof.  / Assoc. Prof.       / Full Prof.

\godina{2023}
\mjesec{Siječanj} % mjesec obrane rada ili projekta

\indeks{0016133479} % broj indeksa ili JMBAG

\smjer{Organizacija poslovnih sustava}
% (ili:
%     Informacijski sustavi, 
%     Poslovni sustavi, 
%     Ekonomika poduzetništva, 
%     Primjena informacijske tehnologije u poslovanju, 
%     Informacijsko i programsko inženjerstvo, 
%     Baze podataka i baze znanja, 
%     Organizacija poslovnih sustava, 
%     Informatika u obrazovanju
% )


\sazetak{Opsega od 100 do 300 riječi. Sažetak upućuje na temu rada, ukratko se iznosi čime se rad bavi, teorijsko-metodološka polazišta, glavne teze i smjer rada te zaključci.}

\kljucneRijeci{riječ; riječ; ...riječ; Obuhvaća $7\pm2$ ključna pojma koji su glavni predmet rasprave u radu.}

\begin{document}

\maketitle

\tableofcontents

\makeatletter \def\@dotsep{4.5} \makeatother
\pagestyle{plain}

\chapter{Uvod}

\chapter{Funkcijsko programiranje}

Funkcijsko programiranje je pristup programiranju koji se temelji na pozivima funkcija kao primarnom konstruktu programiranja.
Općenito pruža praktične pristupe rješavanju problema i pruža uvide u mnoge aspekte računarstva \cite{michaelson2011introduction}.
Posebno, sa svojim korijenima u teoriji računarstva, čini poveznicu između formalnih metoda u računarstvu i njihovu primjenu.

Funkcijsko programiranje kao osnovu koristi $\lambda$-račun što znači da se problemi rješavaju korištenjem
isključivo korištenjem funkcija i njihovih povratnih vrijednosti. Glavna svojstva koja imaju funkcijski
programski jezici su nepromjenjivost stanja varijabli, čistoća funkciija, funkcije višeg reda. Više o glavnim svojstvima
u poglavlju \ref{sec:svojstva}. U funkcijskom programiranju programski kod se sastoji od definicije jedne
ili više funkcija, a izvođenje programa svodi se na izračunavanje funkcijskih izraza \cite{rovzic2016lambda}.

\section{Glavna svojstva funkcijsih programskih jezika} \label{sec:svojstva}

Funkcijsko programiranje nije ograničeno samo na funkcijske programske jezike,
mongi multiparadigmatski jezici omogućavaju funkcijski stil programiranja, no kod takvih je
potrebno poštivati slijedeća svojstva \cite{rovzic2016lambda}:
\begin{itemize}
	\item Nepromjenjivost stanja varijabli; Vrijednost varijable, objekta nije moguće promjeniti
	      nakon inicijalizacije. Ovakav pristup stvara veću sigurnost kod korištenja varijabli jer
	      smo sigurni da se varijabla kroz cijeli tok programa neće promijeniti. Kako bi se kreirale
	      nove varijable potrebno je izvršavanje funkcija koje vraćaju neku vrijednost.
	\item Čiste funkcije; Funkcije koje ne ovise o nikakvim vanjskim varijablama. Koriste jedino vrijednosti
	      koje su prosljeđene direktno u funkciju, te vrijednost prema svojstvu nepromijenjivosti ne može mijenjati.
	      Čiste funkcije garantiraju vraćanje iste vrijednosti prosljeđivanjem istih vrijednostu u funkciju,
	      ovo svojstvo pomaže kod optimizacije jer je funkciju potrebno pokrenuti i izračunati jedanput.
	      Čiste funkcije nemaju popratne posljedice osim kalkulacije razultata. Takav pristup eliminira
	      veliki izvor greška u programu i čini redoslijed izvršavanja nebitnim jer ne postoje popratne
	      posljedice te funkcija može biti evaluirana u bilo kojem trenutku \cite{hughes1989functional}.
	\item Funkcije prvog i višeg reda; Funkcije su tretirane kao objekti prvog reda tj. mogu biti rekurzivne,
        višeg reda, polimorfne \cite{10.1145/72551.72554}. Funkcije višeg reda mogu primati kao argumente
        druge funkcije te mogu vraćati funkcije kao povratne vrijednosti.
\end{itemize}

\chapter{Web poslužitelj}

Web poslužitelj u svojem najosnovnijem obliku isoporučuje klijentima zatražene dokumente preko weba.
Web se sastoji od tri glavne komponente \cite{yeager1996web}:
\begin{enumerate}
    \item Sistem adresa, URL-a (engl. Universal Resource Locators); Putanje koje omogućuju korisnicima
        dohvaćanje gotovo bilo koju vrstu informacija s gotovo bilo kojeg mjesta na internetu
    \item HTTP (engl. HyperText Transfer Protocol); Najčešće korišteni protokol za komunikaciju između
        klijenta i poslužitelja putem weba. Zbog svoje relativne jednostavnosti omogućuje mongim programima
        zajednički rad.
    \item HTML (engl. HyperText Markup Language); Najčešće posluživana vrsta dokumenta na webu te je može
        koristiti svaki web preglednik.
\end{enumerate}

Web poslužitelj s klijentima komunicira razmjenom poruka. Uspostavom veze između poslužitelja i klijenta,
klijent može poslati zahtjev poslužitelju. U slučaju HTTP-a klijent šalje zahtjev s određenom metodom,
ovisno o metodi šalje samo zaglavlje ili zaglavlje i tijelo poruke. HTTP zaglavlja su u tekstualnom obliku.
Poslužitelj, kod primanja zaglavlja, interpretira klijentov zahtijev i pokušava ispuniti zahtijev. Ovisno o
uspjehu ispunjenja zahtijeva poslužitelj šalje povratnu informaciju klijentu, povratna informacija se sastoji
od HTTP zaglavlja i ovisno o zahtijevu, neke HTML datoteke, teksta, slike, strukturiranih podataka.

\chapter{Opis implementacije}

\section{Poglavlje druge razine}

\subsection{Poglavlje treće razine}

\subsubsection{Poglavlje četvrte razine}

\chapter{Tehničke upute}

\section{Upute za oblikovanje izgleda rada}

\textbf{Stranice} se oblikuju korištenjem sljedećih parametara:
\begin{itemize}
	\item veličina i oblik papira je A4, okomito usmjerenje, margine 2,5 cm na svakoj strani;

	\item naslovna stranica rada se ne numerira;

	\item nakon naslovne stranice, sve sljedeće stranice do 1. Poglavlja se numeriraju rimskim brojevima, počevši od i;

	\item od 1. poglavlja nadalje, stranice se numeriraju arapskim brojevima;

	\item broj stranice treba pozicionirati desno 1,25 cm od dna stranice, font Arial 9.
\end{itemize}

\textbf{Tekst} rada je potrebno oblikovati sukladno ovom predlošku, odnosno na sljedeći način:
\begin{itemize}
	\item u pisanju teksta koristite font Arial 11 pt, s proredom 1,5 te razmakom 0 pt prije i razmakom 6 pt poslije odlomka, pri čemu je prvi redak uvučen za 1,25 cm;

	\item u naslovima prve razine „3. Razrada teme“ koristite font Arial 18 pt, podebljano, prijelom stranice (svaki naslov prve razine treba biti na novoj stranici), s proredom 1,5 te razmakom 0 pt prije i razmakom 18 pt poslije odlomka;

	\item u naslovima druge razine „2.1. Naslov“ koristite font Arial 16 pt, podebljano, s proredom 1,5 te razmakom 18 pt prije i razmakom 12 pt poslije odlomka;

	\item u naslovima treće razine „2.1.1. Naslov“ koristite font Arial 14 pt, podebljano, s proredom 1,5 te razmakom 12 pt prije i razmakom 6 pt poslije odlomka;

	\item u naslovima četvrte razine „2.1.1.1. Naslov“ koristite font Arial 12 pt, podebljano, s proredom 1,5 te razmakom 6 pt prije i razmakom 6 pt poslije odlomka;

	\item ostalo značajno isticanje cjelina rada može biti istaknuto podebljanim i kurziv slovima, korištenjem fonta Arial 11 pt.
\end{itemize}


\textbf{Slike} u radu je potrebno oblikovati na sljedeći način:

\begin{itemize}
	\item naziv slike navedite ispod slike uz numeraciju;

	\item za nazive slika koristite iste postavke fonta kao i za tekst, ali stavite naziv slike u centrirani položaj;

	\item za oblikovanje same slike koristite font Arial 9 pt za tekst na slici;
	      ispred same slike umetnite jedan prazan redak (osim ako je slika pozicionirana na početku stranice);

	\item nakon naziva slike ostavite jedan redak prazan (osim ako je naziv slike zadnji redak na stranici);

	\item kod prijeloma stranice treba obratiti posebnu pozornost da naziv slike, izvor i sama slika moraju biti na istoj stranici;

	\item slike je potrebno numerirati redom pojavljivanja u tekstu;

	\item ako je slika preuzeta iz drugog izvora, nakon navođenja naziva slike u zagradi navedite izvor, npr. (autor/autorica, godina);

	\item dozvoljeno je napraviti vlastitu preradu slika, grafikona ili tablica na način da se zadrži isti smisao sadržaja, ali promijeni izgled. I u takvim se slučajevima obavezno u nazivu navodi referenca izvornog djela ovako: ; prema [X]

	\item dozvoljeno je preuzeti samo jednu sliku, grafikon ili tablicu u izvornom obliku iz istog izvora. Za doslovno preuzimanje većeg dijela sadržaja potrebno je ishoditi dozvolu nositelja autorskih prava;

	\item primjer označavanja slike možete vidjeti u nastavku (slika \ref{fig:podjela}).
\end{itemize}

\begin{figure}[]
	\centering
	\includegraphics[width=0.9\textwidth]{slike/slika.png}
	\caption{Podjela investicijskih fondova; preuzeto iz \cite{Russell2010AImodern}}
	\label{fig:podjela}
\end{figure}

\textbf{Tablice} rada je potrebno oblikovati sukladno ovim uputama:
\begin{itemize}
	\item naziv tablice navedite iznad slike;

	\item za nazive tablica koristite iste postavke fonta kao i za tekst, ali stavite naziv tablice u centrirani položaj;

	\item za oblikovanje same tablice koristite font Arial 9 pt za tekst u tablici;

	\item tablice numerirajte redom pojavljivanja u tekstu;

	\item prije naziva tablice umetnite jedan redak prazan (osim ako je naziv tablice prvi redak na stranici);

	\item nakon same tablice umetnite jedan prazan redak (osim ako je tablica pozicionirana na kraju stranice);

	\item kod prijeloma stranice treba obratiti posebnu pozornost da naziv tablice, izvor i sama tablica moraju biti na istoj stranici;

	\item ako je tablica preuzeta iz drugog izvora, nakon navođenja naziva tablice potrebno je navesti izvor, na isti način kako je opisano kod slika;

	\item primjer označavanja tablice možete vidjeti u nastavku (tablica \ref{tab:objekti}).
\end{itemize}

\begin{table}[h!]
	\centering
	\caption{Prikaz podataka o učestalosti pojavljivanja objekta; prema \cite{wooldridge2009IntroductionMultiAgentSystems}}
	\begin{tabularx}{0.66\textwidth}{|X|X|X|X|}
		\hline
		\cellcolor{gray!25} & \cellcolor{gray!25} & \cellcolor{gray!25} & \cellcolor{gray!25} \\
		\hline
		                    &                     &                     &                     \\
		\hline
		                    &                     &                     &                     \\
		\hline
	\end{tabularx}
	\\[10pt]
	\label{tab:objekti}
\end{table}

\textbf{Programski kod}
\begin{itemize}
	\item za oblikovanje teksta koji je programski kôd koristite font Courier, veličine 10 pt, jednostruki prored, 6 pt iza odlomka, npr. HTML kôd dijela zaglavlja početne web stranice FOI weba je prikazan kao isječak koda \ref{lst:prviPrimjer}. U slučaju preuzetog programskog koda, za isti je nužno potrebno naznačiti izvor, kao u isječku koda \ref{lst:kod2}. Ponekad palatali mogu stvarati probleme u opisima isječaka koda, pa ih je potrebno zamijeniti \LaTeX\ kodovima: \v{c}/\v{s}/\v{z}: \lstinline+\v{c/s/z}+, \'{c}: \lstinline+\'{c}+, \dj: \lstinline+\dj+.
\end{itemize}

\begin{lstlisting}[language=HTML, caption={Primjer zapisa isječka programskog kôda}, label=lst:prviPrimjer]
<head>
  <meta http-equiv="Content-Type" content="text/html; charset=utf-8" />
  <link rel="shortcut icon" href="https://www.foi.unizg.hr/sites/default/files/favicon_0_1.ico" type="image/vnd.microsoft.icon" />
  <meta name="generator" content="Drupal 7 (http://drupal.org)" />
  <link rel="canonical" href="https://www.foi.unizg.hr/hr" />
  <link rel="shortlink" href="https://www.foi.unizg.hr/hr" />
  <!-- Set the viewport width to device width for mobile -->
  <meta name="viewport" content="width=device-width, initial-scale=1.0">
  <title>Dobro %*došli*) na FOI | FOI</title>...
</head>
\end{lstlisting}

\begin{lstlisting}[language=Python, caption={
[Ovo je primjer koda koji je preuzet]
Ovo je primjer koda koji je preuzet iz \cite[str. 23]{Russell2010AImodern}}, label=lst:kod2]
print("Ovo je preuzeti dio koda")
\end{lstlisting}

\textbf{Formule}
\begin{itemize}
	\item za unos formula koristite editor za formule. Matematičke izraze moguće je pisati unutar teksta $E = mc^2$ ili izdvojeno:

	      $$
		      a^2 + b^2 = c^2
	      $$
\end{itemize}

\textbf{Kratice}
\begin{itemize}
	\item ako želite koristiti kratice pojmova u tekstu, kad prvi put spominjete pojam potrebno je navesti puni naziv, a kraticu navesti u zagradi: informacijske i komunikacijske tehnologije (IKT). Nakon toga možete koristiti kratice u tekstu. Poželjno je u naslovima koristiti pune nazive.
\end{itemize}

\textbf{Strano nazivlje}
\begin{itemize}
	\item strano nazivlje se u tekstu navodi u zagradi, napisano \textit{kurzivom}, nakon hrvatskog izraza, npr. analiza društvene mreže (engl. \textit{Social Network Analysis - SNA}).
\end{itemize}



\section{Navođenje literature}

Za navođenje literature u radu koristite \textbf{IEEE} stil. Važno je dosljedno primjenjivati odabrani stil u cijelom radu.

U popisu literature potrebno je navesti svu literaturu i samo literaturu koju ste koristili u tekstu.

Uz svaku preuzetu tvrdnju potrebno je navesti njezin izvor, tj. referencu. Reference se u tekstu navode tako da se uz citirani tekst navede izvor, sukladno načinu propisanom odabranim stilom i FOI preporukama za citiranje i referenciranje \cite{sveucilisteuzagrebufakultetorganizacijeiinformatike2017FOIPreporukeCitiranja}. Prilikom referenciranja knjiga, uvijek je potrebno navoditi i broj ili brojeve stranica.

Za citate i njihovo referenciranje koristite naredbu \lstinline+\blockquote+ kako slijedi. Citat će se automatski oblikovati kako je potrebno, ovisno o duljini citata. U kratkom citatu je pojašnjeno da bi se hibridna platforma za orkestraciju mogla u domeni videoigara koristiti
\blockquote[{\cite[str. 5]{schatten2020PlatformeZaOrkestraciju}}]{za implementaciju većeg broja inteligentnijih, a samim
	time i interesantnijih, suparnika.} Ono što slijedi nakon obavezno navedenog broja stranice uz izvor citata je primjer duljeg citata.

\blockquote[{\cite[str. 6]{schatten2020PlatformeZaOrkestraciju}}]{Iako je testiranje performansi MMO igara dobro implementirano na brojnim platformama, testiranje iskustva
	igranja (posebno u primjerice MMORPG igrama) velikog razmjera postaje iznimno složen zadatak, kao što
	je okvirno predstavljeno u (Schatten, Okreša Ðurić, Tomičić i Ivkovič, 2017). Ne samo da je potrebno testirati
	razne logičke zagonetke i zadatke koje igrač mora riješiti kako bi napredovao u igri, već i pojavljujuću interakciju među igračima koja bi mogla u potpunosti promijeniti rezultat igre. Korištenjem distribuirane platforme za orkestraciju bilo bi moguće dodavanje automatiziranih igrača po volji instanciranjem novih agenata. Štoviše, rezultati testiranja mogli bit biti dodatno
	analizirani dodavanjem agenata za analizu ili izvještavanje.}

Podaci o svim bibliografskim jedinicama nalaze se u \lstinline+lib.bib+ datoteci u \lstinline+BibLaTeX+ formatu. Bibliografske jedinice korištene u ovom dokumentu su:
\begin{itemize}
	\item knjige \cite{Russell2010AImodern,wooldridge2009IntroductionMultiAgentSystems,oraictolic2011AkademskoPismoStrategije},
	\item članak s konferencije \cite{okresaduric2019ModellingFormingTemporary},
	\item članci iz časopisa \cite{SchattenEtAl2016roadmap,rincon2017InfluencingPeopleSocial},
	\item web-stranica \cite{copeland2020ArtificialIntelligence}.
\end{itemize}



\chapter{Zaključak}

Ovdje treba sažeto rezimirati najvažnije rezultate razrade teme rada. Potrebno je sažeto opisati što je predmet rada \cite{copeland2020ArtificialIntelligence}, koje su metode, tehnike, programski alati ili aplikacije korištene u razradi rada te koje su pretpostavke dokazane, a koje opovrgnute. Sadržajno, ono što se u uvodu rada najavljuje i kasnije je obuhvaćeno u samom radu, moralo bi biti opisano u zaključnom dijelu kroz rezultate rada.

\lipsum[1-2]

\makebackmatter
% generira popis korištene literature, popis slika (ako je primjenjivo), popis tablica (ako je primjenjivo) i popis isječaka koda (ako je primjenjivo)

\appendices % ako nije potrebno, obrisati ili zakomentirati

\end{document}
